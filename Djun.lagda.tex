\documentclass{article}

\usepackage{newunicodechar}
\usepackage{geometry}[margin=1.25in]
\usepackage{amsmath}
\usepackage{amssymb}
\usepackage{parskip}
% The coloring distracts the author.
\usepackage[bw]{agda}
\usepackage{unicode-math}
\usepackage{physics}
\usepackage{fancyref}
\usepackage[toc]{glossaries}
\usepackage[backend=bibtex]{biblatex}
\usepackage{xcolor}
\usepackage{hyperref}
\usepackage{quiver}
\usepackage{adjustbox}

% What is a good place for this crap?
\newunicodechar{⊤}{\ensuremath{\mathnormal{\top}}}
\newunicodechar{⊥}{\ensuremath{\mathnormal{\bot}}}
\newunicodechar{ℕ}{\ensuremath{\mathnormal{\mathbb{N}}}}
\newunicodechar{₁}{\ensuremath{\mathnormal{_1}}}
\newunicodechar{₂}{\ensuremath{\mathnormal{_2}}}
\newunicodechar{≅}{\ensuremath{\mathnormal{\cong}}}
\newunicodechar{ε}{\ensuremath{\mathnormal{\epsilon}}}
\newunicodechar{τ}{\ensuremath{\mathnormal{\tau}}}
\newunicodechar{λ}{\ensuremath{\mathnormal{\lambda}}}
\newunicodechar{ℚ}{\ensuremath{\mathnormal{\mathbb{Q}}}}
\newunicodechar{ℤ}{\ensuremath{\mathnormal{\mathbb{Z}}}}
\newunicodechar{∷}{\ensuremath{\mathnormal{\Colon}}}
\newunicodechar{⊎}{\ensuremath{\mathnormal{\uplus}}}
\newunicodechar{≈}{\ensuremath{\mathnormal{\approx}}}
\newunicodechar{≉}{\ensuremath{\mathnormal{\napprox}}}
\newunicodechar{≡}{\ensuremath{\mathnormal{\equiv}}}
\newunicodechar{≢}{\ensuremath{\mathnormal{\nequiv}}}
\newunicodechar{≤}{\ensuremath{\mathnormal{\leq}}}
\newunicodechar{⊔}{\ensuremath{\mathnormal{\sqcup}}}
\newunicodechar{≟}{\ensuremath{\mathnormal{\stackrel{?}{=}}}}
\newunicodechar{∘}{\ensuremath{\mathnormal{\circ}}}
\newunicodechar{∧}{\ensuremath{\mathnormal{\land}}}
\newunicodechar{∧}{\ensuremath{\mathnormal{\land}}}
\newunicodechar{⇒}{\ensuremath{\mathnormal{\Rightarrow}}}
\newunicodechar{⟨}{\ensuremath{\mathnormal{\langle}}}
\newunicodechar{⟩}{\ensuremath{\mathnormal{\rangle}}}
\newunicodechar{∎}{\ensuremath{\mathnormal{\blacksquare}}}
\newunicodechar{∈}{\ensuremath{\mathnormal{\in}}}
\newunicodechar{∉}{\ensuremath{\mathnormal{\notin}}}
\newunicodechar{ᵇ}{\ensuremath{\mathnormal{^\AgdaFontStyle{b}}}}
\newunicodechar{∣}{\ensuremath{\mathnormal{\lvert}}}
\newunicodechar{↭}{\ensuremath{\mathnormal{\leftrightsquigarrow}}}

\newcommand{\category}[1]{\mathsf{#1}}

\title{A Brief Look at \(\category{Djun}\), the Knowledge-and-Such Category}
\author{andrew christopher hawk}

\makeglossaries{}

\begin{document}
\maketitle{}

\begin{abstract}
In this paper, the author uses type theory and category theory to discuss the nature of belief and knowledge.  A bit more specifically, the author describes \(\category{Djun}\), which is the knowledge category and has objects like a belief type and a knowledge type.
\end{abstract}

\section{Background}
On 2024-11-24, whilst sitting in a Meeting for Worship, the author reasoned out of an urge to say something.  After such reasoning, the author felt an urge to draw a commutative diagram on the category whose objects include belief, knowledge, and epistemology --- yes, really.\footnote{Determining whether the urge is the action of the divine or schizophrenia is an exercise for the reader.}  Accordingly, the author drew a cheesy and error-filled commutative diagram\footnote{See \fref{fig:originalDiagram}.} on a piece of paper; this paper is effectively a cleaned-up version of that commutative diagram.

\begin{figure}
  \adjustbox{scale=0.75, center}{
    \begin{tikzcd}
      &&&& {\sum_{\left(c : \mathrm{CondB}\right)} \sum_{\left(l : \mathrm{List} B\right)} \left(\mathrm{Supp}\ l\ c\right)} & {\sum_{\left(q : \mathrm{Quant}\right)} \left(\mathrm{List} \left(\mathrm{QB}\ q\right) \times \mathrm{QB}\ q\right)} \\
      &&&& {\mathrm{List}\ B \times B} \\
      & SK & K & {B \times E} & {\mathrm{CondB}} & {\mathrm{CondB+}} \\
      CK \\
      && B & E
      \arrow[curve={height=6pt}, from=1-5, to=3-2]
      \arrow[from=1-6, to=3-6]
      \arrow[from=2-5, to=3-5]
      \arrow[curve={height=6pt}, from=3-2, to=1-5]
      \arrow[from=3-2, to=3-3]
      \arrow[curve={height=6pt}, from=3-3, to=3-4]
      \arrow[curve={height=6pt}, from=3-4, to=3-3]
      \arrow[from=3-4, to=5-3]
      \arrow[from=3-4, to=5-4]
      \arrow[from=3-5, to=5-3]
      \arrow[from=3-5, to=5-4]
      \arrow[from=3-6, to=5-3]
      \arrow[from=4-1, to=3-2]
      \arrow["{?}"{description}, dashed, from=5-4, to=5-3]
    \end{tikzcd}}
  \caption{A digital version of the commutative diagram which inspired this paper.  The author is aware of the jackedness of the typesetting.}\label{fig:originalDiagram}
\end{figure}

\section{Prerequisites for Total Understanding}

\begin{code}
open import Data.Product using (Σ; _×_; proj₁; proj₂)
open import Level using (Level; _⊔_)
open import Data.List using (List)
open import Data.Maybe using (Maybe)
\end{code}

\section{Belief}

\begin{code}
beliefLevel : Level
beliefLevel = {!!}
Belief : Set beliefLevel
Belief = {!!}
\end{code}

\section{Epistemology}

\begin{code}
epistemologyLevel : Level
epistemologyLevel = {!!}
Epistemology : Set epistemologyLevel
Epistemology = {!!}
\end{code}

\section{Knowledge}

\begin{code}
Knowledge : Set (beliefLevel ⊔ epistemologyLevel)
Knowledge = Belief × Epistemology
\end{code}

\begin{code}
knowledge-is-belief : Knowledge → Belief -- K -> B;
knowledge-is-belief = proj₁
\end{code}

\begin{code}
knowledge-demands-an-epistemology :
  Knowledge → Epistemology -- K -> E;
knowledge-demands-an-epistemology = proj₂
\end{code}

\begin{code}
belief-joi-epistemology-is-knowledge :
  Belief → Epistemology → Knowledge -- BxE -> K;
belief-joi-epistemology-is-knowledge = {!!}
\end{code}

\section{Conditional Belief}

\begin{code}
conditionalBeliefLevel : Level
conditionalBeliefLevel = {!!}
record ConditionalBelief : Set (beliefLevel ⊔ conditionalBeliefLevel) where
  Condition : Set conditionalBeliefLevel
  Condition = {!!}
  field
    belief : Belief
    condition : Condition
\end{code}

\begin{code}
conditional-beliefs-are-beliefs : ConditionalBelief → Belief -- CB -> b
conditional-beliefs-are-beliefs = ConditionalBelief.belief
\end{code}

\begin{code}
conditionals-are-like-epistemologies :
  ConditionalBelief → Epistemology
conditionals-are-like-epistemologies = {!!}
\end{code}

\begin{code}
list-to-conditional-belief :
  List Belief × Belief → ConditionalBelief  -- LBxB -> CB;
list-to-conditional-belief = {!!}
\end{code}

\begin{code}
all-belief-can-be-conditional : Belief → ConditionalBelief
all-belief-can-be-conditional = {!!}
\end{code}

\section{The Epistemological Support Type}

\begin{code}
Supp : Epistemology → Belief → Set (epistemologyLevel ⊔ beliefLevel)
Supp = {!!}
\end{code}

\section{Sound Knowledge}

\begin{code}
SoundKnowledge : Set (epistemologyLevel ⊔ beliefLevel)
SoundKnowledge = Σ Knowledge (λ k → Σ Epistemology (λ e → Supp e (proj₁ k)))
\end{code}

\begin{code}
sound-knowledge-is-knowledge : SoundKnowledge → Knowledge -- SK -> K;
sound-knowledge-is-knowledge = {!!}
\end{code}

\section{Correct Knowledge}

\begin{code}
correctKnowledgeLevel : Level
correctKnowledgeLevel = {!!}
record CorrectKnowledge : Set ( epistemologyLevel
                              ⊔ beliefLevel
                              ⊔ correctKnowledgeLevel
                              ) where
  Correct : Knowledge → Set correctKnowledgeLevel
  Correct = {!!}
  field
    sound-knowledge : SoundKnowledge
    said-knowledge-is-correct :
      Correct (sound-knowledge-is-knowledge sound-knowledge)
\end{code}

\begin{code}
correct-knowledge-is-logically-sound :
  CorrectKnowledge → SoundKnowledge -- CK -> SK;
correct-knowledge-is-logically-sound = {!!}
\end{code}

\section{Quantified Belief}

\begin{code}
quantLevel : Level
quantLevel = {!!}
Quant : Set quantLevel
Quant = {!!}
\end{code}

\begin{code}
nullQuantification : Quant
nullQuantification = {!!}
\end{code}

\begin{code}
quantifiedBeliefLevel : Level
quantifiedBeliefLevel = {!!}
QuantifiedBelief : Quant → Set quantifiedBeliefLevel
QuantifiedBelief = {!!}
\end{code}

\begin{code}
nully-quantified-beliefs-are-just-beliefs :
  QuantifiedBelief nullQuantification → Belief -- QB0 -> B
nully-quantified-beliefs-are-just-beliefs = {!!}
\end{code}

\begin{code}
but-are-still-quantified :
  QuantifiedBelief nullQuantification →
  Σ Quant QuantifiedBelief -- QB0 -> QB
but-are-still-quantified = {!!}
\end{code}

\section{Conditional Quantified Belief}

\begin{code}
ConditionalBelief+ : Set (quantLevel ⊔ quantifiedBeliefLevel)
ConditionalBelief+ = Σ Quant (λ q → List (QuantifiedBelief q)
                                  × QuantifiedBelief q)
\end{code}

\begin{code}
quantified-conditional-beliefs-are-beliefs :
  ConditionalBelief+ → Belief -- CBP -> B;
quantified-conditional-beliefs-are-beliefs = {!!}
\end{code}

\begin{code}
quantified-conditionals-are-like-epistemologies :
  ConditionalBelief+ → Epistemology -- CBP -> E;
quantified-conditionals-are-like-epistemologies = {!!}
\end{code}

\begin{code}
beliefs-are-quantified-beliefs :
  Belief → Σ Quant QuantifiedBelief -- B -> QB;
beliefs-are-quantified-beliefs = {!!}
\end{code}

\begin{code}
conditionals-are-quantified-conditionals :
  ConditionalBelief → ConditionalBelief+ -- CB -> CBP;
conditionals-are-quantified-conditionals = {!!}
\end{code}
\end{document}
