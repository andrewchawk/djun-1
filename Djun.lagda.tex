\documentclass{article}

\usepackage{newunicodechar}
\usepackage{geometry}[margin=1.25in]
\usepackage{amsmath}
\usepackage{amssymb}
\usepackage{parskip}
% The coloring distracts the author.
\usepackage[bw]{agda}
\usepackage{unicode-math}
\usepackage{physics}
\usepackage{fancyref}
\usepackage[toc]{glossaries}
\usepackage[backend=bibtex]{biblatex}
\usepackage{xcolor}
\usepackage{hyperref}
\usepackage{quiver}
\usepackage{adjustbox}

% What is a good place for this crap?
\newunicodechar{⊤}{\ensuremath{\mathnormal{\top}}}
\newunicodechar{⊥}{\ensuremath{\mathnormal{\bot}}}
\newunicodechar{ℕ}{\ensuremath{\mathnormal{\mathbb{N}}}}
\newunicodechar{₁}{\ensuremath{\mathnormal{_1}}}
\newunicodechar{₂}{\ensuremath{\mathnormal{_2}}}
\newunicodechar{≅}{\ensuremath{\mathnormal{\cong}}}
\newunicodechar{ε}{\ensuremath{\mathnormal{\epsilon}}}
\newunicodechar{τ}{\ensuremath{\mathnormal{\tau}}}
\newunicodechar{λ}{\ensuremath{\mathnormal{\lambda}}}
\newunicodechar{ℚ}{\ensuremath{\mathnormal{\mathbb{Q}}}}
\newunicodechar{ℤ}{\ensuremath{\mathnormal{\mathbb{Z}}}}
\newunicodechar{∷}{\ensuremath{\mathnormal{\Colon}}}
\newunicodechar{⊎}{\ensuremath{\mathnormal{\uplus}}}
\newunicodechar{≈}{\ensuremath{\mathnormal{\approx}}}
\newunicodechar{≉}{\ensuremath{\mathnormal{\napprox}}}
\newunicodechar{≡}{\ensuremath{\mathnormal{\equiv}}}
\newunicodechar{≢}{\ensuremath{\mathnormal{\nequiv}}}
\newunicodechar{≤}{\ensuremath{\mathnormal{\leq}}}
\newunicodechar{⊔}{\ensuremath{\mathnormal{\sqcup}}}
\newunicodechar{≟}{\ensuremath{\mathnormal{\stackrel{?}{=}}}}
\newunicodechar{∘}{\ensuremath{\mathnormal{\circ}}}
\newunicodechar{∧}{\ensuremath{\mathnormal{\land}}}
\newunicodechar{∧}{\ensuremath{\mathnormal{\land}}}
\newunicodechar{⇒}{\ensuremath{\mathnormal{\Rightarrow}}}
\newunicodechar{⟨}{\ensuremath{\mathnormal{\langle}}}
\newunicodechar{⟩}{\ensuremath{\mathnormal{\rangle}}}
\newunicodechar{∎}{\ensuremath{\mathnormal{\blacksquare}}}
\newunicodechar{∈}{\ensuremath{\mathnormal{\in}}}
\newunicodechar{∉}{\ensuremath{\mathnormal{\notin}}}
\newunicodechar{ᵇ}{\ensuremath{\mathnormal{^\AgdaFontStyle{b}}}}
\newunicodechar{∣}{\ensuremath{\mathnormal{\lvert}}}
\newunicodechar{↭}{\ensuremath{\mathnormal{\leftrightsquigarrow}}}

\newcommand{\category}[1]{\mathsf{#1}}

\title{A Brief Look at \(\category{Djun}\), the Knowledge-and-Such Category}
\author{andrew christopher hawk}

\makeglossaries{}

\begin{document}
\maketitle{}

\begin{abstract}
In this paper, the author uses type theory and category theory to discuss the nature of belief and knowledge.  A bit more specifically, the author describes \(\category{Djun}\), which is the knowledge category and has objects like a belief type and a knowledge type.
\end{abstract}

\section{Background}
On 2024-11-24, whilst sitting in a Meeting for Worship, the author reasoned out of an urge to say something.  After such reasoning, the author felt an urge to draw a commutative diagram on the category whose objects include belief, knowledge, and epistemology --- yes, really.\footnote{Determining whether the urge is the action of the divine or schizophrenia is an exercise for the reader.}  Accordingly, the author drew a cheesy and error-filled commutative diagram\footnote{See \fref{fig:originalDiagram}.} on a piece of paper; this paper is effectively a cleaned-up version of that commutative diagram.

\begin{figure}
  \adjustbox{scale=0.75, center}{
    \begin{tikzcd}
      &&&& {\displaystyle \sum_{\left(c : \mathrm{CondB}\right)\ \left(l : \mathrm{List} B\right)} \left(\mathrm{Supp}\ l\ c\right)} & {\displaystyle \sum_{\left(q : \mathrm{Quant}\right)} \left(\mathrm{List} \left(\mathrm{QB}\ q\right) \times \mathrm{QB}\ q\right)} \\
      &&&& {\mathrm{List}\ B \times B} \\
      & SK & K & {B \times E} & {\mathrm{CondB}} & {\mathrm{CondB+}} \\
      CK \\
      && B & E
      \arrow[curve={height=6pt}, from=1-5, to=3-2]
      \arrow[from=1-6, to=3-6]
      \arrow[from=2-5, to=3-5]
      \arrow[curve={height=6pt}, from=3-2, to=1-5]
      \arrow[from=3-2, to=3-3]
      \arrow[curve={height=6pt}, from=3-3, to=3-4]
      \arrow[curve={height=6pt}, from=3-4, to=3-3]
      \arrow[from=3-4, to=5-3]
      \arrow[from=3-4, to=5-4]
      \arrow[from=3-5, to=5-3]
      \arrow[from=3-5, to=5-4]
      \arrow[from=3-6, to=5-3]
      \arrow[from=4-1, to=3-2]
      \arrow["{?}"{description}, dashed, from=5-4, to=5-3]
    \end{tikzcd}}
  \caption{A digital version of the commutative diagram which inspired this paper.  The author is aware of the jackedness of the typesetting.}\label{fig:originalDiagram}
\end{figure}

\section{Prerequisites for Total Understanding}

\begin{code}
open import Data.Product using (Σ; _×_; proj₁; proj₂; _,_)
open import Level using (Level; _⊔_)
open import Data.List using (List)
open import Data.Maybe using (Maybe)
open import Data.Unit.Polymorphic using (⊤ ; tt)
\end{code}

\section{Belief}
Formally, a term \AgdaBound{t} of type \AgdaRecord{Belief} \AgdaBound{b} is a belief in the existence of a term of type \AgdaField{Belief.belief} \AgdaBound{t}.  Prosaically defining the English term ``belief'' is for now beyond the scope of this paper, but the author hopes that, given the descriptions of the other objects, understanding \AgdaRecord{Belief} will be reasonably simple.

\begin{code}
record Belief (a : Level) : Set (Level.suc a) where
  field
    belief : Set a
\end{code}

\section{Epistemology}

\begin{code}
Epistemology : (e : Level) → Set (Level.suc e)
Epistemology = {!!}
\end{code}

\section{Knowledge}

\begin{code}
Knowledge : (b e : Level) → Set (Level.suc (b ⊔ e))
Knowledge b e = Belief b × Epistemology e
\end{code}

\begin{code}
knowledge-is-belief : {b e : Level} → Knowledge b e → Belief b -- K -> B;
knowledge-is-belief = proj₁
\end{code}

\begin{code}
knowledge-demands-an-epistemology :
  {b e : Level} → Knowledge b e → Epistemology e -- K -> E;
knowledge-demands-an-epistemology = proj₂
\end{code}

\begin{code}
belief-joi-epistemology-is-knowledge :
  {b e : Level} → Belief b → Epistemology e → Knowledge b e -- BxE -> K;
belief-joi-epistemology-is-knowledge b e = b , e
\end{code}

\section{Conditional Belief}

\subsection{Condition}

\begin{code}
Condition : (a : Level) → Set (Level.suc a)
Condition a = Set a

nullCondition : {c : Level} → Condition c
nullCondition = ⊤
\end{code}

\subsection{Conditional Belief}
A bit formally, a term \AgdaBound{c} of type \AgdaRecord{ConditionalBelief} is equivalent to the belief which can be characterized as follows: If there exists a term of type \AgdaField{ConditionalBelief.condition} \AgdaBound{c}, then \AgdaField{ConditionalBelief.belief} \AgdaBound{c} is correct.

\begin{code}
record ConditionalBelief (b c : Level) : Set (Level.suc (b ⊔ c)) where
  field
    belief : Belief b
    condition : Condition c
\end{code}

\subsubsection{Possible Confusion}
\paragraph{``But You Have the Conditional Belief!''}
Regardless of reasonability, whilst not believing that \(B\) holds, simultaneously believing that \(A\) holds \emph{and} that if \(A\) holds, then \(B\) holds is possible.  A bit more formally --- and possibly more coherently ---, one can say that there exists some belief-haver \(x\) such that \(x\) does not believe that \(B\) holds but \emph{does} believe the following things:

\begin{itemize}
  \item If \(A\) holds, then \(B\) holds.
  \item \(A\) holds.
\end{itemize}

\begin{code}
conditional-beliefs-are-beliefs :
  {b c : Level} → ConditionalBelief b c → Belief (b ⊔ c) -- CB -> B;
conditional-beliefs-are-beliefs c = record
  { belief = ConditionalBelief.condition c → Belief.belief (ConditionalBelief.belief c)
  }
\end{code}

\begin{code}
conditionals-are-like-epistemologies :
  {b c : Level} → ConditionalBelief b c → Epistemology b -- CB -> E;
conditionals-are-like-epistemologies = {!!}
\end{code}

\begin{code}
list-to-conditional-belief :
  {b c : Level} →
  List (Belief c) × Belief b →
  ConditionalBelief b c -- LBxB -> CB;
list-to-conditional-belief = {!!}
\end{code}

\begin{code}
all-belief-can-be-conditional : {b c : Level} → Belief b → ConditionalBelief b c
all-belief-can-be-conditional b = record
  { belief = b
  ; condition = nullCondition
  }
\end{code}

\section{The Epistemological Support Type}

\begin{code}
Supp : {b e : Level} → Epistemology e → Belief b → Set (b ⊔ e)
Supp = {!!}
\end{code}

\section{Sound Knowledge}

\begin{code}
SoundKnowledge : (b e : Level) → Set (Level.suc e ⊔ Level.suc b)
SoundKnowledge b e =
  Σ (Knowledge b e × Epistemology e)
    (λ (k , e) → Supp e (proj₁ k))
\end{code}

\begin{code}
sound-knowledge-is-knowledge :
  {b e : Level} → SoundKnowledge b e → Knowledge b e -- SK -> K;
sound-knowledge-is-knowledge = {!!}
\end{code}

\section{Correct Knowledge}

\begin{code}
correctKnowledgeLevel : Level
correctKnowledgeLevel = {!!}
record CorrectKnowledge (b e : Level) :
                        Set ( Level.suc e
                            ⊔ Level.suc b
                            ⊔ correctKnowledgeLevel
                            ) where
  Correct : {b e : Level} → Knowledge b e → Set correctKnowledgeLevel
  Correct = {!!}
  field
    knowledge : Knowledge b e
    said-knowledge-is-correct : Correct knowledge
\end{code}

\begin{code}
correct-knowledge-is-logically-sound :
  {b e : Level} → CorrectKnowledge b e → SoundKnowledge b e -- CK -> SK;
correct-knowledge-is-logically-sound = {!!}
\end{code}

\section{Quantified Belief}

\begin{code}
Quant : (a : Level) → Set a
Quant = {!!}
\end{code}

\begin{code}
nullQuantification : {q : Level} → Quant q
nullQuantification = {!!}
\end{code}

\begin{code}
QuantifiedBelief : (b : Level) → {q : Level} → Quant q → Set (Level.suc (b ⊔ q))
QuantifiedBelief = {!!}
\end{code}

\begin{code}
nully-quantified-beliefs-are-just-beliefs :
  {b c q : Level} → QuantifiedBelief b {q} nullQuantification → Belief b -- QB0 -> B
nully-quantified-beliefs-are-just-beliefs = {!!}
\end{code}

\begin{code}
but-are-still-quantified :
  {b c q : Level} →
  QuantifiedBelief b {q} nullQuantification →
  Σ (Quant q) (QuantifiedBelief b {q}) -- QB0 -> QB
but-are-still-quantified Q = nullQuantification , Q
\end{code}

\begin{code}
actually-quantified-beliefs-are-beliefs-too :
  {b c q : Level} → Σ (Quant q) (QuantifiedBelief b {q}) → Belief (b ⊔ q)
actually-quantified-beliefs-are-beliefs-too = {!!}
\end{code}

\section{Conditional Quantified Belief}

\begin{code}
record ConditionalBelief+ (b q : Level) : Set (Level.suc (b ⊔ q)) where
  field
    quantification : Quant q
    belief : QuantifiedBelief b quantification
    condition : List (QuantifiedBelief b quantification)
\end{code}

\begin{code}
quantified-conditional-beliefs-are-beliefs :
  {b q : Level} → ConditionalBelief+ b q → Belief b -- CBP -> B;
quantified-conditional-beliefs-are-beliefs = {!!}
\end{code}

\begin{code}
quantified-conditionals-are-like-epistemologies :
  {b q : Level} → ConditionalBelief+ b q → Epistemology b -- CBP -> E;
quantified-conditionals-are-like-epistemologies = {!!}
\end{code}

\begin{code}
beliefs-are-quantified-beliefs :
  {b q : Level} → Belief b → Σ (Quant q) (QuantifiedBelief b {q}) -- B -> QB;
beliefs-are-quantified-beliefs = {!!}
\end{code}

\begin{code}
conditionals-are-quantified-conditionals :
  {b c q : Level} → ConditionalBelief b c → ConditionalBelief+ b q -- CB -> CBP;
conditionals-are-quantified-conditionals = {!!}
\end{code}
\end{document}
